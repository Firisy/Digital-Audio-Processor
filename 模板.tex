\documentclass{article}
\usepackage{ctex}
\usepackage[a4paper, left=2.5cm, right=2.5cm, top=2.54cm, bottom=2.54cm]{geometry}
\usepackage{graphicx}
\usepackage{caption}
\usepackage{subcaption}
\usepackage{multirow}
\usepackage{multicol}
\usepackage{booktabs}
\usepackage[normalem]{ulem}
\usepackage{eqparbox}
\usepackage{listings}
\usepackage{mdframed}
\usepackage{amsmath}
\usepackage{enumerate}
\usepackage{placeins}
\usepackage{float}
\usepackage[dvipsnames]{xcolor}  % 使用 xcolor 包的 dvipsnames 选项
\usepackage[table,xcdraw]{xcolor}
% \usepackage{minted}
\useunder{\uline}{\ul}{}
\definecolor{code_bgc}{RGB}{255, 253, 253}  % 定义一个非常淡的粉色
\lstset{
    numbers=left,                   % 在左侧显示行号
    numberstyle=\color{Lavender},  % 设置行号的样式
    % frame=single,                   % 添加框架
    rulecolor=\color{black},        % 设置框架的颜色
    tabsize=2,                      % 设置tab的宽度
    breaklines=true,                % 自动折行
    breakatwhitespace=false,        % 在空白处折行
    escapeinside={\%*}{*)},         % 如果你想在代码中添加LaTeX代码,可以在此处设置
    keywordstyle=\color{RedViolet},      % 设置关键字的颜色
    commentstyle=\color{Salmon},   % 设置注释的颜色
    stringstyle=\color{RoyalBlue},      % 设置字符串的颜色
    basicstyle=\footnotesize,       % 设置代码的字体大小
    columns=fullflexible, 
    xleftmargin=2em,                % 设置左边距
    backgroundcolor=\color{code_bgc}, % 设置背景颜色
}
\newenvironment{codedisplay}[2]
{
    \begin{multicols}{3}
        \lstinputlisting[language={#1}]{#2}
    \end{multicols}
    
}

\begin{document}

\begin{table}
    \begin{tabular}{clllll}
    \multicolumn{4}{c}{\multirow{3}{*}{\includegraphics[width=4.5cm]{assets/image.png}
    \fontsize{30}{35}\selectfont\kaishu 实验报告}}            & \quad 专业: & {\ul{\eqparbox{col4}{\quad 电子信息工程 \quad}}}     \\
    \multicolumn{4}{c}{}                                      & \quad 姓名: & {\ul {\eqparbox{col4}{\qquad 冯静怡}}}        \\
    \multicolumn{4}{c}{}                                      & \quad 学号: & {\ul {\eqparbox{col4}{\quad 3220104119}}} \\
    课程名称: & {\ul {\eqparbox{col1}{\quad 电路与电子技术实验II\quad}}} & \quad 指导老师:    & {\ul {\eqparbox{col2}{张伟\quad}}} 
     & \quad 地点: & {\ul {\eqparbox{col4}{ 紫金港东三406}}}   \\
    实验名称: & {\ul {\eqparbox{col1}{\quad 数字钟\quad }}}       &\quad 同组学生: & {\ul {\eqparbox{col2}{陈亦乔\quad} }}
    &\quad 日期: & {\ul {\eqparbox{col4}{\today}}}      
    \end{tabular}
    \end{table}

\section{实验目的}
\begin{enumerate}
    \item 学会使用VHDL语言描述数字电路;
    \item 学会使用FPGA编写基本数字电路;
    \item 掌握Quartus软件的基本用法。
\end{enumerate}

\end{document}